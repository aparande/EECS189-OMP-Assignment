\documentclass[11pt,letterpaper]{article}
\usepackage[lmargin=1in,rmargin=1in,tmargin=1in,bmargin=1in]{geometry}

% -------------------
% Packages
% -------------------
\usepackage{
	amsmath,			% Math Environments
	amssymb,			% Extended Symbols
	enumerate,		    % Enumerate Environments
	graphicx,			% Include Images
	lastpage,			% Reference Lastpage
	multicol,			% Use Multi-columns
	multirow			% Use Multi-rows
}


% -------------------
% Font
% -------------------
\usepackage[T1]{fontenc}
\usepackage{charter}


% -------------------
% Heading Commands
% -------------------
\newcommand{\class}{EECS 16ML}
\newcommand{\term}{Fall 2020}
\newcommand{\instructor}{Team RAAAK}
\newcommand{\head}[2]{%
\thispagestyle{empty}
\vspace*{-0.5in}
\noindent\begin{tabular*}{\textwidth}{l @{\extracolsep{\fill}} r @{\extracolsep{6pt}} l}
	\textbf{Quiz Topic: Outlier Removal via OMP} \\
	\textbf{\class:\; \term} & & \\
	\textbf{\instructor}
\end{tabular*} \\
\rule[2ex]{\textwidth}{2pt} %
}


% -------------------
% Commands
% -------------------
\newcounter{problem}
\newcommand{\problem}{
	\stepcounter{problem}%
	\noindent \textbf{Problem \theproblem. }%
}
\newcommand{\pointproblem}[1]{
	\stepcounter{problem}%
	\noindent \textbf{Problem \theproblem.} (#1 points)\,%
}
\newcommand{\pspace}{\par\vspace{\baselineskip}}
\newcommand{\ds}{\displaystyle}


% -------------------
% Header & Footer
% -------------------
\usepackage{fancyhdr}

\fancypagestyle{pages}{
	%Headers
	\fancyhead[L]{}
	\fancyhead[C]{}
	\fancyhead[R]{}
\renewcommand{\headrulewidth}{0pt}
	%Footers
	\fancyfoot[L]{}
	\fancyfoot[C]{}
	\fancyfoot[R]{}
\renewcommand{\footrulewidth}{0.0pt}
}
\headheight=0pt
\footskip=14pt

\pagestyle{pages}


% -------------------
% Content
% -------------------
\begin{document}
\head{}


% Question 1
\problem Use the following matrix equation setup to run OMP for two iterations. Please box the intermediate and final residuals as well as the two components identified to have non-zero entries. For notational uniformity, denote each of the columns of 
$\begin{bmatrix}
        \boldsymbol{A} & \boldsymbol{I}
\end{bmatrix}$ 
as $\vec{c_{1}}$, $\vec{c_{2}}$, ..., $\vec{c_{8}}$.

\begin{equation*}
    \begin{bmatrix}
        \boldsymbol{A} & \boldsymbol{I}
    \end{bmatrix}\begin{bmatrix}
        \vec{x} \\
        \vec{f}
    \end{bmatrix}
     = \vec{y} \Rightarrow
    \begin{bmatrix}
         1 & 1 & 0 & 1 & 1 & 0 & 0 & 0 \\
         1 & 1 & 0 & 0 & 0 & 1 & 0 & 0 \\
         1 & 0 & 1 & 1 & 0 & 0 & 1 & 0 \\
         1 & 0 & 1 & 0 & 0 & 0 & 0 & 1
    \end{bmatrix}
    \begin{bmatrix}
         a \\ b \\ c \\ d \\ e \\ f \\ g \\ h
    \end{bmatrix}
    =\begin{bmatrix}
         2 \\ 0 \\ 2 \\ 1
    \end{bmatrix}
\end{equation*}
\vspace{9cm}

% Question 2
\problem Ignore the calculations done in the first problem. Suppose that a genie ran OMP for the problem above and told you the following about the sparse solution: %\vfill
\begin{equation*}
    \vec{x} = \begin{bmatrix}
        \frac{1}{2} \\ 1 \\ -\frac{1}{2} \\ 1
    \end{bmatrix}, \vec{f} = \begin{bmatrix}
        0 \\ 10 \\ -\frac{1}{2} \\ 0
    \end{bmatrix}
\end{equation*}
Interpret the results by identifying the outlier(s). Provide justification/explanation.
\newpage


% Question 3
\problem In each of the three parts below, describe a potential stopping condition discussed in this course for OMP. In addition to naming the stopping condition, describe potential (dis)advantages and/or use cases. The order you list them in does not matter.
	\begin{itemize}
	\item Stopping Condition 1: \vspace{3cm}
	\item Stopping Condition 2: \vspace{3cm}
	\item Stopping Condition 3: \vspace{3cm}
	\end{itemize} \vspace{6cm}





\end{document}
