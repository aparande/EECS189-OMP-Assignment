\documentclass[12pt]{article}

\setlength{\topmargin} {-1in}

% Style modifications
\oddsidemargin  0.25in
\evensidemargin 0.25in
\textwidth      6.0in
%\textheight     8.0in
%\textheight     8.5in % Changed on 13 September 2012
\textheight     9in % Changed on 3 October 2013
\parskip        0.1in
\parindent      0.0in
\headheight     1.0in

\headsep        .25in


%\pagestyle      {headings}

% Packages
\usepackage{graphicx}
\usepackage{amsmath,amsfonts,amssymb,amscd,verbatim,graphicx,fancyhdr,tikz}
\usepackage{circuitikz}
\usepackage{mathtools} %for minus plus sign alignment in matrices and vectors, added 13 Feb 2017
\usepackage{pgfplots}
\usetikzlibrary{arrows,automata}
\usetikzlibrary{positioning}
\usepgfplotslibrary{groupplots}
\pgfplotsset{compat=1.16}

\usepackage{palatino}
%\usepackage{cmbright}
\usepackage{enumerate}
\usepackage{multicol}
\usepackage{listings}
\usepackage{color}
\usepackage{float}
\usepackage{esdiff} % Derivatives
\usepackage{cleveref}
\usepackage{siunitx}
\usepackage{caption}
\usepackage{subcaption}
\usepackage[ruled,vlined]{algorithm2e}

\usepackage{tikz} %Can be used to draw Markov Chains or other Graphs
% See http://www.freeminded.org/index.php/2010/05/a-markov-chain-in-tikz/
\usetikzlibrary{arrows.meta}

\newcommand\eqnnumber{\addtocounter{equation}{1}\tag{\theequation}} % Allows us to tag the last equation in an align


%The accents package is included so I can type bold dots for time derivatives
\usepackage{accents}
\newcommand*{\dt}[1]{%
\accentset{\mbox{\large\bfseries .}}{#1}}
\newcommand*{\ddt}[1]{%
\accentset{\mbox{\large\bfseries .\hspace{-0.25ex}.}}{#1}}



%%% Some Mathematical Environmental Definitions
\newcommand{\reals}{{\rm Reals}}
\newcommand{\B}[1]{\mathbf{#1}}
% the following line was added for the vector bold.
\DeclareMathAlphabet{\V}{OML}{cmm}{b}{it}
% Since the above gives a strange looking bold 0 (zero) we do this:
% (To be used only in math mode)
%\newcommand{\0}{\mbox{\bf 0}}
\newcommand{\bs}[1]{\boldsymbol{#1}}
\newcommand{\underscore}{\char`\_}
\newcommand{\tildetext}{\char`\~}
\newcommand{\defn}{\stackrel{\triangle}{=}}
\newcommand{\cas}{\text{cas }}

\newcommand{\var}{\textup{var}}
\def\P{{\bf P}}
\def\E{{\bf E}}
\newcommand{\cov}{\textup{cov}}

\def\E{{\bf E}}
\def\P{{\bf P}}
\def\var{\text{var}}
\def\cov{\text{cov}}
\def\e{\mathrm{e}}

\DeclareMathOperator*{\argmin}{argmin}
\DeclareMathOperator*{\argmax}{argmax}

\Crefname{algocf}{Algorithm}{Algorithms}

%%% ENTER COURSE NUMBER AND NAME HERE
\newcommand{\coursenumber}{EECS~16ML}
\newcommand{\coursename}{Practical Machine Learning}
\newcommand{\coursetitle}{\coursenumber: \coursename}
\newcommand{\docauthor}{Anmol Parande}

%%% WHAT KIND OF DOCUMENT IS THIS?
\newcommand{\documenttype}{note} % Other examples: exam, quiz, final, midterm, homework
\newcommand{\titledocumenttype}{Note} % This is what appears in the document header

% What is the document number? e.g., 5, if this is homework 5, or quiz 5, ...
\newcommand{\documentnumber}{1}

%%% Problem Prefix, e.g., HW1, or MT2, or F1
\newcommand{\pp}{\documentinitials\documentnumber}

%%% Document Headers
\lhead{{\coursetitle} \\ \titledocumenttype \mbox{} \documentnumber \\ [-10pt]}
\rhead{Fall 2020 \\ [-10pt]}

